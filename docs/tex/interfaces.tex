\chapter{Interfaces} \label{interfaces}

\section{AHB-Lite Interface}\label{ahb-lite-interface}

The AHB-Lite interface is a regular AHB-Lite slave port. All signals are
supported. See the \emph{AMBA 3 AHB-Lite Specification} for a complete
description of the signals.

\begin{longtable}[]{@{}lccl@{}}
\toprule
Port & Size & Direction & Description\tabularnewline
\midrule
\endhead
HRESETn & 1 & Input & Asynchronous active low reset\tabularnewline
HCLK & 1 & Input & Clock Input\tabularnewline
HSEL & 1 & Input & Bus Select\tabularnewline
HTRANS & 2 & Input & Transfer Type\tabularnewline
HADDR & HADDR\_SIZE & Input & Address Bus\tabularnewline
HWDATA & HDATA\_SIZE & Input & Write Data Bus\tabularnewline
HRDATA & HDATA\_SIZE & Output & Read Data Bus\tabularnewline
HWRITE & 1 & Input & Write Select\tabularnewline
HSIZE & 3 & Input & Transfer Size\tabularnewline
HBURST & 3 & Input & Transfer Burst Size\tabularnewline
HPROT & 4 & Input & Transfer Protection Level\tabularnewline
HREADYOUT & 1 & Output & Transfer Ready Output\tabularnewline
HREADY & 1 & Input & Transfer Ready Input\tabularnewline
HRESP & 1 & Output & Transfer Response\tabularnewline
\bottomrule
\caption{AHB-Lite Interface
	Ports}
\end{longtable}

\subsection{HRESETn}\label{hresetn}

When the active low asynchronous HRESETn input is asserted (`0'), the
interface is put into its initial reset state.

\subsection{HCLK}\label{hclk}

HCLK is the interface system clock. All internal logic for the AMB3-Lite
interface operates at the rising edge of this system clock and AHB bus
timings are related to the rising edge of HCLK.

\subsection{HSEL}\label{hsel}

The AHB-Lite interface only responds to other signals on its bus -- with
the exception of the global asynchronous reset signal HRESETn -- when
HSEL is asserted (`1'). When HSEL is negated (`0') the interface
considers the bus IDLE.

\subsection{HTRANS}\label{htrans}

HTRANS indicates the type of the current transfer.

\begin{longtable}[]{@{}clp{10cm}@{}}
\toprule
HTRANS & Type & Description\tabularnewline
\midrule
\endhead
00 & IDLE & No transfer required\tabularnewline
01 & BUSY & Connected master is not ready to accept data, but intents to
continue the current burst.\tabularnewline
10 & NONSEQ & First transfer of a burst or a single
transfer\tabularnewline
11 & SEQ & Remaining transfers of a burst\tabularnewline
\bottomrule
\caption{AHB-Lite Transfer Type (HTRANS)}
\end{longtable}

\subsection{HADDR}\label{haddr}

HADDR is the address bus. Its size is determined by the HADDR\_SIZE
parameter and is driven to the connected peripheral.

\subsection{HWDATA}\label{hwdata}

HWDATA is the write data bus. Its size is determined by the HDATA\_SIZE
parameter and is driven to the connected peripheral.

\subsection{HRDATA}\label{hrdata}

HRDATA is the read data bus. Its size is determined by HDATA\_SIZE
parameter and is sourced by the APB4 peripheral.

\subsection{HWRITE}\label{hwrite}

HWRITE is the read/write signal. HWRITE asserted (`1') indicates a write
transfer.

\subsection{HSIZE}\label{hsize}

HSIZE indicates the size of the current transfer.

\begin{longtable}[]{@{}cll@{}}
\toprule
HSIZE & Size & Description\tabularnewline
\midrule
\endhead
000 & 8bit & Byte\tabularnewline
001 & 16bit & Half Word\tabularnewline
010 & 32bit & Word\tabularnewline
011 & 64bits & Double Word\tabularnewline
100 & 128bit &\tabularnewline
101 & 256bit &\tabularnewline
110 & 512bit &\tabularnewline
111 & 1024bit &\tabularnewline
\bottomrule
\caption{Transfer Size Values
	(HSIZE)}
\end{longtable}

\subsection{HBURST}\label{hburst}

HBURST indicates the transaction burst type -- a single transfer or part
of a burst.

\begin{longtable}[]{@{}cll@{}}
\toprule
HBURST & Type & Description\tabularnewline
\midrule
\endhead
000 & SINGLE & Single access\tabularnewline
001 & INCR & Continuous incremental burst\tabularnewline
010 & WRAP4 & 4-beat wrapping burst\tabularnewline
011 & INCR4 & 4-beat incrementing burst\tabularnewline
100 & WRAP8 & 8-beat wrapping burst\tabularnewline
101 & INCR8 & 8-beat incrementing burst\tabularnewline
110 & WRAP16 & 16-beat wrapping burst\tabularnewline
111 & INCR16 & 16-beat incrementing burst\tabularnewline
\bottomrule
\caption{AHB-Lite Burst Types (HBURST)}
\end{longtable}

\subsection{HPROT}\label{hprot}

The HPROT signals provide additional information about the bus transfer
and are intended to implement a level of protection.

\begin{longtable}[]{@{}ccl@{}}
\toprule
Bit\# & Value & Description\tabularnewline
\midrule
\endhead
3 & 1 & Cacheable region addressed\tabularnewline
& 0 & Non-cacheable region addressed\tabularnewline
2 & 1 & Bufferable\tabularnewline
& 0 & Non-bufferable\tabularnewline
1 & 1 & Privileged Access\tabularnewline
& 0 & User Access\tabularnewline
0 & 1 & Data Access\tabularnewline
& 0 & Opcode fetch\tabularnewline
\bottomrule
\caption{Protection Signals (HPROT)}
\end{longtable}

\subsection{HREADYOUT}\label{hreadyout}

HREADYOUT indicates that the current transfer has finished. Note, for
the AHB-Lite Timer this signal is constantly asserted as the core is
always ready for data access.

\subsection{HREADY}\label{hready}

HREADY indicates whether or not the addressed peripheral is ready to
transfer data. When HREADY is negated (`0') the peripheral is not ready,
forcing wait states. When HREADY is asserted (`1') the peripheral is
ready and the transfer completed.

\subsection{HRESP}\label{hresp}

HRESP is the instruction transfer response and indicates OKAY (`0') or
ERROR (`1').

\section{Timer Interface}\label{timer-interface}

\subsection{TIMER\_INTERRUPT}\label{timer_interrupt}

TIMER\_INTERRUPT is a single output signal that is asserted the
following conditions are both met:

\begin{enumerate}
\def\labelenumi{\arabic{enumi}.}
\item
  Any bit of the IPENDING register is asserted
\item
  The corresponding bit of the IENABLE register is also asserted.
\end{enumerate}

This may also be written as:

TIMER\_INTERRUPT \textless{}= IPENDING \& IENABLE

