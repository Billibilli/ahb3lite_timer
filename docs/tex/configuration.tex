\chapter{Configuration} \label{configurations}

\section{Introduction}\label{introduction-1}

The size and implementation style of the timer module is defined via HDL
parameters as specified below.

\section{Core Parameters}\label{core-parameters}

\begin{longtable}[]{@{}lccl@{}}
\toprule
Parameter & Type & Default & Description\tabularnewline
\midrule
\endhead
TIMERS & Integer & 3 & Number of Timers\tabularnewline
HADDR\_SIZE & Integer & 32 & Width of AHB-Lite Address
Bus\tabularnewline
HDATA\_SIZE & Integer & 32 & Width of AHB-Lite Data Buses\tabularnewline
\bottomrule
\end{longtable}

\protect\hypertarget{_Toc326677729}{}{}Table 3‑1: Core Parameters

\subsection{TIMERS}\label{timers}

The parameter TIMERS defines the number of timers supported and thereby
the number of TIMECMP registers implemented by the core. Values between
1 and 32 are supported, with the default defined as `3'.

\subsection{HADDR\_SIZE}\label{haddr_size}

The HADDR\_SIZE parameter specifies the address bus size to connect to
the AHB-Lite based host.

\subsection{HDATA\_SIZE}\label{hdata_size}

The HDATA\_SIZE parameter specifies the data bus size to connect to the
AHB-Lite based host. The maximum size supported is 64 bits.

\section{Core Registers}\label{core-registers}

\begin{longtable}[]{@{}lllll@{}}
\toprule
Register & Address & Size & Access & Function\tabularnewline
\midrule
\endhead
PRESCALER & Base + 0x00 & 32bits & Read/Write & Timebase\tabularnewline
IPENDING & Base + 0x08 & 32bits & Read Only & Interrupt
Pending\tabularnewline
IENABLE & Base + 0x0C & 32bits & Read/Write & Interrupt
Enable\tabularnewline
TIME & Base + 0x10 & 64bits & Read/Write & Timer Register\tabularnewline
TIMECMP[n] & Base + 0x18+8n & 64bits & Read/Write & Compare
Value\tabularnewline
\bottomrule
\end{longtable}

Note: `n' represents an integer for 0 to TIMERS-1.

\subsection{PRESCALER}\label{prescaler}

The Timer module operates synchronously with the AHB-Lite bus clock
input HCLK. A 32 bit PRESCALER register enables the time base for the
timers to be less than that of HCLK by dividing this clock frequency by
the value of PRESCALER + 1.

For example: If PRESCALER=3, the timer will increment every PRESCALE+1=4
cycles of HCLK, setting the time base to HCLK/4 Hz.

The default value of PRESCALER=0, thereby setting the timer clock
frequency equal to the bus (HCLK) frequency. The TIME counter starts
incrementing once the register PRESCALER is written to for the first
time (See section 3.2.4).

Note: The value of PRESCALER value can only be defined once after the
peripheral is released from reset.

\subsection{IPENDING}\label{ipending}

IPENDING is a 32-bit read-only register that indicates if a timer
interrupt is pending.

Each bit of the IPENDING register corresponds to one timer with the
position of each bit indicating the associated timer. E.g. bit zero
indicates the interrupt status of Timer[0]. IPENDING bits associated
with unimplemented timers are tied low (`0')

An interrupt pending bit is set when the value of TIMECMP[n] equals
the value of TIME. It is cleared by a write to the associated
TIMECMP[n] register, as specified in the RISC-V privileged spec
1.9.1.

\subsection{IENABLE}\label{ienable}

IENABLE is a 32-bit Read/Write register, where each bit of the register
is a dedicated 'Interrupt Enable' bit for each time. The bit position
indicates the associated timer. E.g. Interrupt Enable for Timer[0]
is located at bit position 0.

Only TIMERS bits are implemented with the remaining MSBs always read as
'0'. A write to the unused MSBs has no effect.

An interrupt is generated when a bit of IPENDING is set and its
associated IENABLE bit is also set. This allows the core to be used in
(1) pure POLL mode, where the CPU polls the status of the bits to
determine if a timer event happened, (2) pure interrupt driven mode,
where each timer can generate an interrupt, or (3) a combination of the
above.

\subsection{TIME}\label{time}

The TIME register is a common 64-bit high-resolution time-keeping
counter used by all timers. It is the basis for the RDCYCLE instruction
as specified in the RISC-V privileged spec 1.9.1 and may be written to
also in accordance with the RISC-V specification.

The time base for the TIME register is derived from the AHB-Lite bus
clock HCLK, as described in section 3.2.1, and is defined as:

Freq\textsubscript{TIME} = Freq\textsubscript{HCLK} / (PRESCALER+1)

The counter starts incrementing once the register PRESCALER is written
to for the first time.

\subsection{TIMECMP[n]}\label{timecmpn}

For each timer (as defined by the parameter TIMER) there is a dedicated
64 bit Time Compare register which defines when the IPENDING bits are
asserted

These registers are denoted as TIMECMP[n], where `n' is an index
from 0 to TIMERS-1, and are located consecutively in the address space
according to the formula:

Base Address of TIMECMP[n] = 0x18 + 8n

For example, TIMECMP[0] is located at address 0x18, TIMECMP[1]
at 0x20, TIMECMP[1] at 0x28 etc.

The IPENDING bit associated with the TIMECMP register is set when the
TIMECMP[n] value equals the value of TIME.

IPENDING[n] = (TIMECMP[n] == TIME)

Writing the TIMECMP[n] register clears bit `n' of the IPENDING
register.

